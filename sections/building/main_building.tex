% !TEX program = lualatex
\documentclass[../../main.tex]{subfiles}
\begin{document}

\section{Building}%
\label{sec:building}

A preliminary examination of using yocto versus using the mender-convert utility can be seen
in table \ref{tab:build_comp}. For reference the raspbian buster lite image is approximately
1.8GB in size.


\begin{table}[h]
	\centering
	\caption{
		Build comparison of the default mender raspberrypi yocto build and a mender-convert build
		of the default raspbian buster light image.
	}
	\label{tab:build_comp}
	\begin{tabular}{l|ccc}
		& \textbf{Yocto with sstate caching}& \textbf{Yocto} & \textbf{Mender-convert}\\
		\hline
		\textbf{Build time}&3m43.789s&83m11.750s& $\sim$ 20m \\
		\textbf{Image size}&604MB&604MB&7.67GB\\
	\end{tabular}
\end{table}
\todo{Add numbers to table buidl comparison}


For creating images mender offers 2 possibilities.
The first is to user the mender-convert utility, which converts a debian image
to one with mender support. This approach has little influence on any workflow which relies
on generating a debian image.\\
The second option is to build the image using the yocto project and including menders yocto layers.
To properly understand the advantages and disadvantages of using yocto, here is a short introduction.

\subsection{Yocto}%
\label{sub:yocto}

Yocto is build system for creating linux distributions for embedded devices. It relies on meta-layers
each modifying the resulting image. These meta-layers can contain board specific settings,
applications, modifications to other layers and so forth.



\subsubsection{Layers}%
\label{ssub:layers}

The default layers for mender-raspberrypi
\begin{itemize}
	\item meta-poky
	\item meta-yocto-bsp
	\item meta-oe
	\item meta-python
	\item meta-networking
	\item meta-multimedia
	\item meta-raspberrypi
	\item meta-mender-core
	\item meta-mender-raspberrypi
	\item meta-mender-demo
\end{itemize}
Though it should build without the mender demo layer.

\subfile{yocto_layers.tex}




\end{document}
