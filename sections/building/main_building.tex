% !TEX program = lualatex
\documentclass[../../main.tex]{subfiles}
\begin{document}

\section{Building}%
\label{sec:building}

\begin{table}[h]
	\centering
	\caption{Build comparison of Mender yocto build and mender-convert build}
	\label{tab:build_comp}
	\begin{tabular}{l|cc}
		& \textbf{Yocto} & \textbf{Mender-convert}\\
		\hline
		\textbf{Build time}&83m11.750s& $\sim$ 20m \\
		\textbf{Image size}&604MB&7.67GB\\
		\textbf{Delta updates}&&
	\end{tabular}
\end{table}
\todo{Add numbers to table buidl comparison}


For creating images mender offers 2 posiibilities.
The first is to user the mender-convert utility, which converts a debian image
to one with mender support. This approach has little influence on any workflow which relies
on generating a debian image.\\
The second option is to build the image using the yocto project and including menders yocto layer.
To proberly understand the advantages and disadvanteges of using yocto, here is a short introduction.

\subsection{Yocto}%
\label{sub:yocto}

Yocto is build system for creating linux distributions for embedded devices. It relies on meta-layers
each modifying the resulting image. These meta-layers can contain board specific settings,
applications, modifications to other layers and so forth.









\end{document}
