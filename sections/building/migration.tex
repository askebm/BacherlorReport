% !TEX program = lualatex
\documentclass[../../main.tex]{subfiles}
\begin{document}



The primary applicatins of the MCLURS project reside in this
\href{https://gitlab.com/esrl/mclurs.git }{gitlab} project.


\paragraph{Steps}%
\label{par:steps}


\begin{itemize}
	\item \mintinline{bash}{devtool add https://gitlab.com/esrl/mclurs.git}
\end{itemize}
This creates a workspace workspace folder like this:

\dirtree{%
	.1 workspace.
	.2 recipes.
	.2 sources.
	.2 $\cdots$.
}
\mintinline{bash}{devtool} will automatically attempt to create a recipe that builds the project.
The mclurs git however does not have a \mintinline{bash}{Makefile} in the root of the repository.
Instead the make files are located in \mintinline{bash}{mclurs-1.0/} and
\mintinline{bash}{libmclurs-snap-perl-1.0/}.
The most correct thing to do here is to have separate recipes for each sub-project, ideally
dividing them into separate git repositories.
As to not disturb any current workflow an extra \mintinline{bash}{Makefile} is provided, that
builds and installs the sub-projects. This also allow the use of the
\mintinline{bash}{autotools.bbclass} which provides default pre-compile and compile functions.
All that is left to do is to provide a install function and adding the files to the package which can be seen in listing \ref{lst:mclurs_install}.

\begin{listing}
\begin{minted}{bitbake}
do_install () {
	oe_runmake install DESTDIR=${D} SBINDIR=${sbindir} MANDIR=${mandir} INCLUDEDIR=${includedir}
}
FILES_${PN} = "${sbindir}/* ${bindir}/* ${datadir}/*"
\end{minted}
\caption{do\_install and package files for mclurs.}
\label{lst:mclurs_install}
\end{listing}

Normally when creating a perl recipe it usally enough to inherit from the
\mintinline{bash}{cpan.bbclass} to setup the correct build environment, but as
\mintinline{bash}{libmclurs-snap-perl-1/} only has an install target and no compiling
is not needed, using a normal \mintinline{bash}{Makefile} suffices here.\\

There is also a custom version of the usbduxfast kernel module which the mclurs system
make use of. It resides in \mintinline{bash}{mclurs-1.0/adc/kernel/}. Usually the
\mintinline{bash}{usbduxfast.h} file is replaced in the linux kernel is replaced with this custom
one written by John Hallam.
There 2 ways to go about this; either create a patch or prepend an action to the pre-compile
function that replaces the usbduxfast module.
Both solutions are viable, but the most commonly accepted solution is to make a patch.
An easy way to do this is again to use \mintinline{bash}{devtool}.
\mintinline{bash}{devtool modify linux-raspberrypi} will populate
\mintinline{bash}{<WORKDIR>/build/workspace/sources/linux-raspberrypi/ }
with the source from the linux-kernel used for the raspberrypi.
Then its just replacing the usbduxfast file, git committing the change and
using \mintinline{bash}{devtool finish -m patch linux-raspberrypi <WORKDIR>/meta-mclurs}.
This generate a \mintinline{bash}{.bbappend} file in the meta-mclurs layer and a patch file
that will then be applied.\\

The mclurs project is runtime dependent on zsh, daemontools, runit and some perl modules.
There is already recipe for both zsh and daemontools, but not for runit.
The trouble with runit is that is an init system would replace systemd that mender is dependent on.
The best solution would be to either port mender to runit or mclurs to systemd.
As a middle of the road alternative a runit recipe that runs as a systemd service is created using
the same workflow as the mclurs project. The recipe is based on 
\href{https://aur.archlinux.org/cgit/aur.git/tree/PKGBUILD?h=runit-systemd}{this}
AUR makepkg file.\\

The \mintinline{bash}{libmclurs-snap-perl-1.0} has a perl dependency tree which can be seen in
figure \ref{fig:libmclurs perl dep}


\begin{figure}[h]
	\centering
	\usetikzlibrary {graphs,graphdrawing} \usegdlibrary {layered}
	\begin{tikzpicture}[build/.style={draw,color=red}, run/.style={draw,color=blue}]
		\graph [layered layout, edges=rounded corners]
		{
			"libmclurs-snap-perl-1.0"[draw] -> {
				"ZMQ -"[run] -> {
					"Sub::Name +"[run],
					"ZMQ::Constants -"[run] -> "Module::Install -"[build],
					"ZMQ::LibZMQ3 -"[run] -> {
						"Task::Weaken +"[run],
						"Devel::CheckLib -"[build],
						"Test::Requires +"[build],
						"Test::Fatal +"[build] -> "Try::Tiny +"[build],
						"Test::TCP +"[build] -> "Test::SharedFork +"[build],
						"Module::Install -"
					}
				},
				"ZMQ::LibZMQ3 -"
			}
		};
	\matrix [draw,below left] at (current bounding box.north east) {
		\node [build,label=right:Build dependencies] {}; \\
		\node [run,label=right:Runtime dependencies] {}; \\
		\node [label=right:Available recipe] {+}; \\
		\node [label=right:No available recipe] {-}; \\
	};
	\end{tikzpicture}
	\caption{Perl dependency tree for libmclurs-snap-perl-1.0.}%
	\label{fig:libmclurs perl dep}
\end{figure}

The dependencies are gathered from \href{https://cpandeps.grinnz.com/}{this} site.

Unfortunately it was not possible to port
\href{https://metacpan.org/pod/ZMQ::LibZMQ3}{ZMQ::LibZMQ3} to yocto mostly due to lack
of understanding of perl build system.\\

The resulting MCLURS layer can be found \href{https://github.com/askebm/meta-mclurs}{github}.
The packages provides by the MCLURS layer are listed her:
\begin{itemize}
	\item devel-checklib-perl
	\item file-remove-perl
	\item mclurs (not fully functional in runtime)
	\item module-corelist-perl
	\item module-install-perl
	\item module-scandeps-perl
	\item runit-systemd
	\item yaml-tiny-perl
	\item zeromq
	\item zmq-constants-perl
	\item zmq-libzmq3-perl (not working)
	\item	zmq-perl (not working)
\end{itemize}





\subsubsection{Easier migration option}%
\label{ssub:easier_migration_option}

Instead of creating all these recipes mender has an update module that supports docker
passing docker containers. The value here that not as much dependency resolving is necessary.

Docker would fill all the requirements regarding sandboxing hardware access and so on,
but the cost would be a greater overhead while running the applications, which in turn
means a greater power consumption.

If the application that runs on the deployed device changes frequently a container solution
would make a lot of sense, but as this is not the case with the mclurs system this solution
is suboptimal, but easier from a migration stand point.
An example of a MCLURS docker container can be seen en section \ref{sub:docker}



\end{document}
