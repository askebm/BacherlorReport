% !TEX program = lualatex
\documentclass[../../main.tex]{subfiles}
\begin{document}


\subsubsection{Yocto layers}%
\label{ssub:yocto_layers}

\dirtree{%
	.1 meta-mclur.
	.2 conf.
	.3 layer.conf.
	.2 COPYING.MIT.
	.2 README.
	.2 recipes-kernel.
	.3 linux.
	.4 linux-raspberrypi.
	.5 0001-Canged-usbduxfast.patch.
	.4 linux-raspberrypi\_\%.bbappend.
}

Each layer contains recipes e.g. .bb and .bbappend, these recipes contains instruction for
building packages, which can the be included in the final image
using the \texttt{IMAGE\_INSTALL} variable.


\subsubsection{Migrating to yocto}%
\label{ssub:migrating_to_yocto}

The MCLUR project uses a custom kernel versino of the usbduxfast kernel driver.

\mint{sh}|devtool modify linux-rasberrypi|
\mint{sh}|<replace usbduxfast.c>|
\mint{sh}|devtool finish -m patch linux-rasberrypi ../meta-mclur|

This will generate a patch and a correctly named \texttt{.bbappend} file that will
apply the patch to the rpi-kernel.

\subsubsection{Workflow}%
\label{ssub:workflow}

\dirtree{%
	.1 project.
	.2 build.
	.3 conf.
	.4 bblayer.conf.
	.4 local.conf.
	.2 upstream-sources.
	.3 meta-layers.
	.3 $\vdots$.
	.2 meta-mclur.
	.3 conf.
	.4 layer.conf.
	.3 recipes-kernel.
	.4 linux.
	.5 linux-raspberrypi.
	.6 0001-Canged-usbduxfast.patch.
	.5 linux-raspberrypi\_\%.bbappend.
}



\end{document}
