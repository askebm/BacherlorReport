% !TEX program = lualatex
\documentclass[../../main.tex]{subfiles}
\begin{document}

\subsection{Provisioning a new device}%
\label{sub:provisioning_a_new_device}
Assuming rpi3b+ as to always just run PXE boot. Though if using an older rpi unit use 
\mintinline{bash}{bootbin.sh} to setup an sdcard up with the bootcode.bin file which
enables older rpis to PXE boot.

PXE boot server is hosted on an rpi. A new device will boot a default raspbian buster image
which mounts an nfs share from the development host which have the current sdcard image that 
should be deployed. On the same share there is a script, \mintinline{bash}{boot.sh}, which is
run using a cron job during startup.
This is illustrated in figure \ref{fig:boot-process}.\\




\begin{figure}[h]
\begin{center}
	\begin{tikzpicture}[spring layout,node distance=2cm,node/.style={draw}]
		\node (1)[node] {New Device};
		\node (2)[node] {PXEBOOT server};
		\node (3)[node] {Switch};
		\node (4)[node] {Build Host};
		\draw (1) edge (3)
				(4) edge (3)
				(1) edge["1.",bend left,->,color=blue] (2)
				(2) edge["2.",bend left,->,color=red] (1)
				(1) edge["3.",->,bend right,color=blue!60!black] (4)
				(4) edge["4.",bend right,->,color=red!60!black] (1)
				(3) edge (2);
\end{tikzpicture}
\end{center}
\caption{1. PXEBOOT; 2. Boot default RPI image; 3. Mount NFS share;
	4. Run \mintinline{bash}{boot.sh}  script}%
	\label{fig:boot-process}
\end{figure}




When a device boots the mender image for the first time it will show up as pending on the mender
host interface as can be seen in figure \ref{fig:mender-ui-pending}.
Here it can be accepted as a authorized device.
This is under the assumption that local network on which the developer host attached is secure.


\begin{figure}[h]
	\centering
	\includegraphics[width=1\linewidth]{img/mender-ui-pending-edited.pdf}
	\caption{Mender interface when device is pending}%
	\label{fig:mender-ui-pending}
\end{figure}

\subsection{Updating existing device}%
\label{sub:updating_existing_device}

Updating a device is taken of by the mender server. When compiling a sdcard image using yocto,
a mender artefact is also generated, this can be uploaded to the mender server.
The artefact can then be assigned devices or groups, where any device in an assigned group will
receive the update the next time the deployed device polls the mender server.\\

To ensure robustness mender makes use of what they call an A/B rootfs.
Of rootfs A is version 1.0 adn B is 2.0, then B is in use if it has been committed.
Then if the device updated A will be overridden to the new version and the deployed device will
try to commit the new rootfs.

An overview of the 9 possible states a m,ender client can be seen in figure \ref{fig:mender-states}.
A mender artifact can both be a update for a mender update module or a rootfs update.
With this system mender can deliver robust and reliable updates for deployed devices.

\begin{figure}[h]
\begin{center}
\begin{tikzpicture}[simple necklace layout,node distance=3cm%
	,node/.style={draw}]
	\node[node,color=green!50!black] (1){Idle};
	\node[node] (2){Sync};
	\node[node] (3){Download};
	\node[node] (4){ArtifactInstall};
	\node[node] (5){ArtifactReboot};
	\node[node] (6){ArtifactRollback};
	\node[node] (7){ArtifactRollbackReboot};
	\node[node] (8){ArtifactFailure};
	\node[node] (9){ArtifactCommit};
	\draw (1) edge[->,bend right] (2)
				(2) edge[->,bend right] (3)
				(3) edge[->,bend right] (4)
				(4) edge[->,bend right] (5)
				(5) edge[->,bend right] (6)
				(6) edge[->,bend right] (7)
				(7) edge[->,bend right] (8)
				(8) edge[->,bend left] (1)
				(2) edge[->,bend right] (1)
				(4) edge[->,bend right] (8)
				(5) edge[->,bend right] (9)
				(9) edge[->,bend right] (1)
				(9) edge[->,bend right] (6)
				(6) edge[->,bend left] (8);
\end{tikzpicture}
\end{center}
\caption{Mender states during update}%
\label{fig:mender-states}
\end{figure}

This system with states also allow for great customizability when creating update modules.



\subsection{Tasks}%
\label{sub:tasks}

Tasks are deployed using the mender update module and deploying a task have been made idempotent
to not insure that a device will not try to execute the same task twice over.

A mender update module is an executable located in \texttt{/usr/share/mender/modules/v3}
which takes 2 arguments, state and files.

For this project we assume that any needed program is existing on the deployed device,
so a update module is made that adds tasks as cron jobs.

Tasks are visible in the mender inventory.

\subsection{Docker}%
\label{sub:docker}

Instead of porting projects mender already has a docker update module. This of course reduce
the requirement for porting project at the expense of greater overhead, connected with running
containers.

The custom kernel module is still needed on the deployed device, but other than that it only needs
the docker service.

A mclurs docker file can be seen in listing \ref{lst:mclurs-docker}.

\begin{listing}[h]
\inputminted{docker}{/home/aske/Bachelor/DockerMCLURS/Dockerfile}
\caption{A docker file that enables the MCLURS project to run in a dockerdocker  container.}
\label{lst:mclurs-docker}
\end{listing}

The docker container can then be build and pushed to dockerhub.
A mender artifact can then be generated as shown in listing \ref{lst:docker-mender-push}.
The artefact can be uploaded the mender server and from there a deployment can be created
to any device matching \mintinline{bash}|${DEVICE_TYPE}|.

\begin{listing}[h]
\begin{minted}{bash}
docker build . -t mclurs
docker push ${ID}/${REPO}:mclurs
docker-artifact-gen -n ${ARTIFACT_NAME} -t ${DEVICE_TYPE} -o ${OUTPUT_PATH} ${DOCKER_IMAGES}
\end{minted}
\caption{Generate mender artefact for the docker mender update module.}
\label{lst:docker-mender-push}
\end{listing}





Demo docker with c-sense-hat interacting with hardware.

\end{document}
