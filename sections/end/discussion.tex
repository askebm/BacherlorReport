% !TEX program = lualatex
\documentclass[../../main.tex]{subfiles}
\begin{document}

The solution found here using mender and the yocto project to provide a feasible IoT edge cloud
solution does indeed work. Though it has a greater footprint then complete custom solution but that
is to be expected for a more generic solution. Using a large open source project  also benefits
from more auditing from its users compared to a smaller custom solution.
It should be expected that a larger project \underline{should} be more stable and secure.
Of course one could argue for security by obscurity, but that is generally frowned upon and would
only work as long as the project is very small.


The current project is not as easy to maintain, as the previous distribution method, mainly
because the previous solution was not connected to the internet. The chosen solution does incur
some more maintenance compared converting a debian image.
All the benefits of using the yocto project are not apparent in this project. Going forward
it would be easier for MCLURS system to branch out to use other devices and even use crossover
SoCs with multiple hardware architectures to get more specialised sound recordings.



Migrating to the yocto project is not penalty free, as it requires some more specialised knowledge
about how yocto does things and the multitude of ways a single thing can be achieved.
To help accomadte migration the yocto project has their \texttt{devtool}, which can automate 
most of the migration and create the initial recipes. It doesn't do everything and knowledge about
how the yocto project operates is still needed.

As seen in this project close familiarity with the software being ported to yocto is a huge benefit
as dealing with new workflows and new build systems can make for a confusing debugging experience.

An advantage that choosing the mender project is that they offer to host the mender server, which
can reduce needed maintenance hours in maintaining server security.





\end{document}
