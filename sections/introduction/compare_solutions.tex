% !TEX program = lualatex
\documentclass[../../main.tex]{subfiles}
\begin{document}

2 existing IoT solution for Linux embedded devices are considered.
mender\cite{mender-home} and pantacor\cite{panatahub}.\\

Both solutions support fallback in case an update fails.
They both let the deployed device poll for updates as to not have open ports on the device.

\paragraph{Pantacor}%
\label{par:pantacor}

is written in C and makes use of LXC containers to deploy application, which is also written in C,
giving the entire project an overall small footprint rootfs wise.\\
At the time of writing there are still several todos in their documentation. As exciting as it
would be to contribute to this open source IoT solution, it is out of the scope of this project.


\paragraph{Mender}%
\label{par:mender}

is a lot more mature than pantacor and has the advantage that it can transform a debian image
to a mender image with a fallback image, so this solution can be appended to any already 
existing workflow that makes use of debian images. This process as easy as it is, creates a
rather large 8GB image.\\

Another consideration is that mender supports the yocto project which is a huge project that is
used to create custom embedded Linux distributions. This seems perfect to facilitate the use of
a patched kernel.\\

\subsubsection{Yocto}%
\label{ssub:yocto}

Migrating to the yocto project also brings other less obvious advantages.
If the mender project unexpectedly should cease to exist another solution could
be implemented in a new layer.
Yocto is also a very stable project with both Cisco and Microsoft Azure using it it.
\\


Based on the lacking maturity of pantacor and the fact that mender supports yocto, mender is
chosen.

\end{document}
