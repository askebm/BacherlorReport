% !TEX program = lualatex
\documentclass[../../main.tex]{subfiles}
\begin{document}

As this project aims to implement an IoT like workflow, for deploying sandboxed applications
on Linux embedded devices. In this project only raspberrypis are used, but with some
tooling it should be made to work on many other devices.\\

There already exists open-source solutions which can be taken advantage of, as it is generally a bad
idea to write security yourself.

To make the project more goal oriented it aims to port the MCLURS project to this workflow.
This is done both to prove the workflow works, but also to delve deep into the workflow
to discover any quirks that might make this workflow unsuitable.

The report is divided into 3 main sections, building, deployment and security.

Building covers everything up until using the system to deploy devices and tasks.
Deployment describes how device are provisioned, deployed and given tasks.
Security covers which security concerns have been taken into account.

To further specify the project a series of requirements are listed in section \ref{sub:requirements}.


\subsection{Existing solution}%
\label{sub:existing_solution}
\subfile{existing_solution.tex}

\subsection{Requirements}%
\label{sub:requirements}
\subfile{requirements.tex}

\subsection{Comparing existing IoT solutions}%
\label{sub:comparing_existing_iot_solutions}
\subfile{compare_solutions.tex}
	
\end{document}
