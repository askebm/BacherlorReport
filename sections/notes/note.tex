% !TEX program = lualatex
\documentclass[../../main.tex]{subfiles}

\subsection*{Mender}%
\label{sub:mender}

Can be created from default debian images, so its less intrusive to
existing workflow
Images as a result of this is huge (requires 8 GB)
https://docs.mender.io/2.2/getting-started/quickstart-with-raspberry-pi
support Yocto as part of their layers, which should reduce image footprint
Will require a new workflow, toolchain, etc.
Huge community in both Yocto and Mender

\subsection*{Pantahub}%
\label{sub:pantahub}

Written in C and make use of LXC containers which are also written in C giving
the complete project an overall very small footprint
Less mature, at time of writing there still several todo in their documantation.
Also a smaller community.

\subsection*{Yocto}%
\label{sub:yocto}

Uses layers, makes building images for specific usecases easier as necessary layers are included.
Greater set of requirements, for host build system, Arch linux not working.
Allows for delta updates in mender to reduce bandwith in Over-The-Air  updates.\\
Makes it easier to transistion hardware.\\
Cross-Prelink: - Generate link tables for dynamic linking.







\subsection*{Tests}%
\label{sub:tests}

\begin{itemize}
	\item Build time\\
		Yocto vs Mender convert
	\item Image size\\
		Yocto vs Mender convert
\end{itemize}


