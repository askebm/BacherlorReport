% !TEX program = lualatex
\documentclass[../../main.tex]{subfiles}
\begin{document}





To generally reduce the risk of a deployed device being hijacked, no ports are open and 
all communication is initiated by the deployed device polling the  mender server for any possible
updates.

The first attack vector that is considered is tampering with device, where an 
unsolicited 3rd party obtains physical access to the deployed device and
transmit false data back.\\

The second attack vector that is considered is spoofing, where in an unsolicited 3rd
party pretends to an already deployed device.\\

Lastly we consider a situation where an attacker have gained control over the
internet connection and is trying to incorporate the deployed device into a botnet.


\subsection{Tamper protection}%
\label{sub:tamper_protection}

Assuming the deployed device is in a box, the box can be wired in such a ways that any opening of
box will trigger an event on the device.

This will not prevent tampering, but will allow server side detection of tampering.
The tampering will marked in the health statistic of the device, which will be encrypted with a
public key from the server.


\subsection{Sign returned data with solokey}%
\label{sub:sign_returned_data_with_solokey}

Assuming and encryption chip being located on the deploy device. This is being emulated with a
solokey connected via USB.

Verify hardware has not been replaced

\subsection{Certificates and keys}%
\label{sub:certificates_and_keys}

Mender handles this.

Should handle if some has control over our internet connection, and tries to spoof being the
mender server.



	
\end{document}
