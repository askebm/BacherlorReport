% !TEX program = lualatex
\documentclass[../../main.tex]{subfiles}
\begin{document}

As soon as a device has an active internet connection security is a concern.
The primary concern is the device being hijacked and used in a botnet for malicious 
purposes.

For the MCLURS project it is also a concern to verify a device such that it is not possible for
anyone to deploy a device and report research data. This also includes that it should not be possible
to tamper with a device, at least not undetected.

Most security regarding the deployed devices are handled either by upstream as it is a bad idea to
do security yourself.
The only concern not covered by upstream is tamper protection/detection and theft.

Theft is not covered here.


%To generally reduce the risk of a deployed device being hijacked, no ports are open and 
%all communication is initiated by the deployed device polling the  mender server for any possible
%updates.
%
%The first attack vector that is considered is tampering with device, where an 
%unsolicited 3rd party obtains physical access to the deployed device and
%transmit false data back.\\
%
%The second attack vector that is considered is spoofing, where in an unsolicited 3rd
%party pretends to an already deployed device.\\
%
%Lastly we consider a situation where an attacker have gained control over the
%internet connection and is trying to incorporate the deployed device into a botnet.


\subsection{Tamper protection}%
\label{sub:tamper_protection}

Assuming the deployed device is in a box, the box can be wired in such a ways that any opening of
box will trigger an event on the device.

This will not prevent tampering, but will allow server side detection of tampering.

To demo this effect a small C program is written that utilises the button on the sense hat
module.

When the event i triggered a time stamp overwrites the current time stamp.
Time stamp is then available in the mender inventory.
When this message then changes in the mender inventory it can then be verified
whether access to the deployed device was legitimate or not.

An example returned from the mender API can be seen in listing \ref{lst:inv-tamp-json}.

\begin{listing}
\begin{minted}[bgcolor=white!0]{json}
{
  "id": "5ed373e8bacd430001464145",
  "attributes": [
    {
      "name": "rootfs_type",
      "value": "ext4"
    }, {
      "name": "mac_eth0",
      "value": "b8:27:eb:c5:de:49"
    }, {
      "name": "ipv4_eth0",
      "value": "192.168.0.41/24"
    }, {
      "name": "device_type",
      "value": "raspberrypi3"
    }, {
      "name": "tamper_date",
      "value": "2020-05-31T14:42:09.02Z"
    }, {
      "name": "mender_client_version",
      "value": "2.2.0"
    }
  ],
  "updated_ts": "2020-05-31T14:42:09.02Z"
}

\end{minted}
\caption{Mender device inventory example in JSON format.}
\label{lst:inv-tamp-json}
\end{listing}



\subsection{Over-The-Air updates}%
\label{sub:over_the_air_updates}

Mender handles OTA via device polling, as in the device polls the mender server for updates.
This ensures that no port on the deployed device needs to be open, greatly lowering the risk of
a deployed device being hijacked.
The device verifies the mender server using SSL certificates.
This should overcome any man in the middle attack.

To ensure robustness mender makes use of what they call an A/B rootfs.
Of rootfs A is version 1.0 adn B is 2.0, then B is in use if it has been committed.
Then if the device updated A will be overridden to the new version and the deployed device will
try to commit the new rootfs.

An overview of the 9 possible states a m,ender client can be seen in figure \ref{fig:mender-states}.
A mender artifact can both be a update for a mender update module or a rootfs update.
With this system mender can deliver robust and reliable updates for deployed devices.

\begin{figure}[h]
\begin{center}
\begin{tikzpicture}[random seed=4, spring layout,node distance=3cm%
	,node/.style={draw}]
	\node[node,color=green!50!black] (1){Idle};
	\node[node] (2){Sync};
	\node[node] (3){Download};
	\node[node] (4){ArtifactInstall};
	\node[node] (5){ArtifactReboot};
	\node[node] (6){ArtifactRollback};
	\node[node] (7){ArtifactRollbackReboot};
	\node[node] (8){ArtifactFailure};
	\node[node] (9){ArtifactCommit};
	\draw (1) edge[->,bend left] (2)
				(2) edge[->,bend left] (3)
				(3) edge[->,bend left] (4)
				(4) edge[->,bend left] (5)
				(5) edge[->,bend left] (6)
				(6) edge[->,bend left] (7)
				(7) edge[->,bend left] (8)
				(8) edge[->,bend left] (1)
				(2) edge[->,bend left] (1)
				(4) edge[->,bend left] (8)
				(5) edge[->,bend left] (9)
				(9) edge[->,bend left] (1)
				(9) edge[->,bend left] (6)
				(6) edge[->,bend left] (8);
\end{tikzpicture}
\end{center}
\caption{Mender states during update}%
\label{fig:mender-states}
\end{figure}
This system with states also allow for great customizability when creating update modules.



%\subsection{Possible extension of signingkeys}%
%\label{ssub:possible_extension_of_signingkey}
%
%Extend mclurs to cross-over SoC and use FIDO2
%
%
%
	
\end{document}
