% !TEX program = lualatex
\documentclass[../../main.tex]{subfiles}
\begin{document}

As soon as a device has an active internet connection security is a concern.
The primary concern is the device being hijacked and used in a botnet for malicious 
purposes.

For the MCLURS project it is also a concern to verify a device such that it is not possible for
anyone to deploy a device and report research data. This also includes that it should not be possible
to tamper with a device, at least not undetected.


Theft is also a concern, but is not covered here.



%To generally reduce the risk of a deployed device being hijacked, no ports are open and 
%all communication is initiated by the deployed device polling the  mender server for any possible
%updates.
%
%The first attack vector that is considered is tampering with device, where an 
%unsolicited 3rd party obtains physical access to the deployed device and
%transmit false data back.\\
%
%The second attack vector that is considered is spoofing, where in an unsolicited 3rd
%party pretends to an already deployed device.\\
%
%Lastly we consider a situation where an attacker have gained control over the
%internet connection and is trying to incorporate the deployed device into a botnet.


\subsection{Tamper protection}%
\label{sub:tamper_protection}

Assuming the deployed device is in a box, the box can be wired in such a ways that any opening of
box will trigger an event on the device.

This will not prevent tampering, but will allow server side detection of tampering.

To demo this effect a small C program is written that utilises the button on the sense hat
module.


\subsection{Certificates and keys}%
\label{sub:certificates_and_keys}

Mender handles this.

Should handle if some has control over our internet connection, and tries to spoof being the
mender server.

ssl security


\subsection{Possible extension of signingkeys}%
\label{ssub:possible_extension_of_signingkey}

Extend mclurs to cross-over SoC and use FIDO2



	
\end{document}
